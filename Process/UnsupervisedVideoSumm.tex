
\documentclass[conference]{IEEEtran}
\usepackage[utf8]{inputenc}
\usepackage{amsmath}
\usepackage{graphicx}
\usepackage{booktabs}
\usepackage{float}
\usepackage{array}
\usepackage[table,xcdraw]{xcolor}
\usepackage{hyperref}
\usepackage{caption}

\title{\Huge SVC Classification Report}
\author{\IEEEauthorblockN{Papagrigoriou Vasileios Savvas}
\IEEEauthorblockA{Auth}}
\date{09-11-2023}

\begin{document}

\maketitle

\section*{Overview}
Unsupervised Multi-Modal Video Summarization is a cutting-edge project at the intersection of computer vision and machine learning. Its primary aim is to automate the process of condensing lengthy videos into succinct summaries without the need for pre-labeled training data, leveraging advancements in object detection and unsupervised learning algorithms.

\section*{Objectives}
\begin{itemize}
    \item Provide an efficient means of summarizing long videos.
    \item Utilize state-of-the-art object detection techniques in understanding video content.
    \item Employ unsupervised learning methods for the automatic generation of video summaries.
\end{itemize}

\section*{Methodology}
\subsection*{Object Detection}
\textbf{Tool Used:} YoloV3, renowned for its accuracy and speed in real-time object detection.\\
\textbf{Process:} Analyzes video frames to identify and categorize various objects.\\
\textbf{Output:} A comprehensive list of detected objects, including their types and coordinates within each frame.

\subsection*{Video Summarization}
\textbf{Dataset:} Incorporates insights from the TVsum50 dataset, a widely recognized benchmark in the video summarization field.\\
\textbf{Approach:} Uses patterns and trends identified in TVsum50 to determine the most critical segments of a video.

\section*{Implementation}
\subsection*{objectDetection.py}
\textbf{Function:} Processes input videos to detect objects using YoloV3.\\
\textbf{Location:} Script reads videos from input\_videos/ and outputs data to be used by the summarization module.

\subsection*{videoSum.ipynb}
\textbf{Core:} Central component for the video summarization task.\\
\textbf{Integration:} Utilizes output from objectDetection.py along with insights from TVsum50 to create video summaries.\\
\textbf{Features:} Includes interactive elements and visualizations for a better understanding of the summarization process.

\section*{Expected Results}
\begin{itemize}
    \item \textbf{Efficiency:} Ability to process various video types, providing quick and accurate summaries.
    \item \textbf{Effectiveness:} Summaries that capture the essence of the original content, focusing on key events and objects.
    \item \textbf{User Engagement:} Improved user experience by providing concise yet comprehensive summaries.
\end{itemize}


\subsection*{Advanced Methodological Insights}

This section delves deeper into the core of the project, highlighting the intricate details and the thought process behind various methodological choices.

\subsubsection*{The Objective of Automated Trailer Generation}
\begin{itemize}
    \item The primary aim is to automate the generation of a concise trailer from a longer video using unsupervised learning, with a focus on KNN algorithms.
    \item This process is underpinned by the YoloV3 model for image object detection and the integration of multiple features like visual and audio components.
\end{itemize}

\subsubsection*{Integration of Visual and Audio Features}
\begin{itemize}
    \item Visual Features: Extracted using the VGG16 model for each frame of the video.
    \item Audio Features: Derived using the MFCC algorithm to capture the audio essence of the video.
    \item The incorporation of LDA (Linear Discriminant Analysis) was experimented with but yielded inferior results compared to using only PCA (Principal Component Analysis).
    \item PCA is employed not only for dimensionality reduction but also to visualize data clusters for better understanding and selection.
\end{itemize}

\subsubsection*{Object Encoding and Clustering}
\begin{itemize}
    \item Object Detection: Objects in video frames are detected and encoded using binary encoding (e.g., 001, 010, 100) to facilitate distinction in the KNN algorithm.
    \item One Hot Encoding was tested alongside binary encoding, resulting in similar outcomes.
    \item K-Means Clustering: The method of determining the number of clusters (n\_clusters) in K-Means is crucial. Currently, the project uses the square root of the square root of the number of features, which, while not the most efficient, provides better performance than methods with larger separations (i.e., more clusters).
\end{itemize}

\subsubsection*{Challenges in Frame Selection and Video Summarization}
\begin{itemize}
    \item The selection of frames for creating a summarized video is still an area of exploration. The current method involves choosing frames from each cluster and creating a 15-second video based on their importance and the knapsack algorithm.
    \item The best video summary is identified by comparing each generated summary with the ground truth and selecting the one with the highest F-score (macro).
    \item A challenge arises in videos without ground\_truth, as there's no ground truth for comparison, making the selection process more subjective.
\end{itemize}


\subsection*{Best Results and Performance Metrics}
\begin{table}[H]

\begin{tabular}{|c|c|c|c|c|c|c|c|}
\hline
\textbf{videoID} & \textbf{Threshold} & \textbf{Precision} & \textbf{Recall} & \textbf{F1 Binary}  & \textbf{Prop\_high\_importance} \\
\hline
8 & 3 & 0.821229 & 0.7 & 0.755784  & 0.821229 \\
\hline
2 & 3 & 0.167939 & 0.104762 & 0.129032  & 0.167939 \\
\hline
4 & 2 & 1 & 0.0971698 & 0.177128  & 1 \\
\hline
6 & 2 & 0.437778 & 0.0619497 & 0.10854 & 0.437778 \\
\hline
9 & 2 & 1 & 0.0327044 & 0.0633374  & 1 \\
\hline
0 & 1 & 1 & 0.0231481 & 0.0452489  & 1 \\
\hline
5 & 2 & 0.12 & 0.0169811 & 0.0297521 & 0.12 \\
\hline
7 & 1 & 1 & 0.0157697 & 0.0310497  & 1 \\
\hline
10 & 2 & 0.192771 & 0.00503145 & 0.00980693  & 0.192771 \\
\hline
1 & 2 & 0.00888889 & 0.00125786 & 0.00220386 & 0.00888889 \\
\hline
3 & 0 & 0 & 0 & 0 & 0  \\
\hline
\end{tabular}
\caption{Performance Metrics for Various Videos}
\end{table}

\begin{table}[H]
\begin{tabular}{|c|c|c|c|c|c|c|c|}
\hline
\textbf{videoID} & \textbf{Threshold} & \textbf{Precision} & \textbf{Recall} & \textbf{F1 Macro} & \textbf{Prop\_high\_importance} \\
\hline
 8&3 &  0.821229 & 0.7 &        0.87061  &             0.821229   \\
\hline
\end{tabular}
\caption{Performance Metrics using Macro (Only first result)}
\end{table}


\subsection*{Further Considerations and Future Research}
Discussion: 27/11/23
\begin{itemize}
    \item Reinforcement Learning: There is consideration to train a model with the current data using Reinforcement Learning to predict ground truths. However, the limited size of the TVSum dataset (50 videos) may impede the effectiveness of this approach.
    \item Frame Selection and Cluster Determination: The project is actively exploring efficient techniques for optimal frame selection and cluster determination, especially in the absence of annotations.
    \item Best Number of Clusters: Finding the ideal number of clusters in K-Means clustering is an ongoing challenge.
    \item Alternative Models: Exploring other models, such as SVM, for improved results on unknown videos is under consideration.
\end{itemize}



\end{document}